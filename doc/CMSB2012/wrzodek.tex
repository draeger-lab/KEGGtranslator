\documentclass[citeauthoryear]{llncs}
\usepackage{graphicx}
%
\begin{document}

\title{Identification of Disease Associated SNPs Depending on Pathway and Protein-Interaction~Data}
\titlerunning{<Your abbreviated contribution title>}

\author{Finja B\"uchel\inst{1} \and Florian Mittag\inst{1} \and Clemens Wrzodek\inst{1}
\and Claudia Schulte\inst{2} \and Manu~Sharma\inst{2} 
\and Thomas Gasser\inst{2} \and Andreas Zell\inst{1}}
\authorrunning{<abbreviated author list>}

\institute{Center for Bioinformatics Tuebingen (ZBIT), University of Tuebingen, T\"ubingen, Germany
\and
Center of Neurology and Hertie-Institute for Clinical Brain Research, University~of~Tuebingen, T\"ubingen, Germany
}

\maketitle


\begin{abstract}
%problem statement:
%      what problem are you going to solve?
%
%motivation/relevance:
%      why is it important to solve this problem?
%
%approach/method:
%      how did you go about solving the problem?
%
%results:
%      what is your solution to the problem?
%
%conclusions:
%      what are the implications of your solution?
\begin{itemize}
\item problem statement: Identification of associated disease specific SNP sets\\
\item up to now the potential utility of GWAS not used\\
\item identification of associated SNPs computational challenging\\
\item We developed a new method by applying a priori biological knowledge (pathway, interaction, known and predicted)\\
\item Statistical evaluation with variation of Fishers method\\
\item Tested the method on PD data.
\end{itemize}

\end{abstract}

\section{Introduction}
%Opening statement
%    Problem: what problem are you going to solve?
%    Relevance: why is it important to solve this problem?
%
%Literature review:
%    Summarize the current state of knowledge in the area of
%              investigation
%    What previous research has been done on the problem?
%    Mark the gap in the current state of knowledge:
%    What piece of knowledge is still lacking in the area of
%              investigation?
%
%Brief overview of your own study:
%    Approach/method: how did you go about filling this gap?
%    Results: what did you find out?
%    Conclusion: what do these results mean (why are these
%               results significant?

\begin{itemize}
\item Problem: GWAS is expensive, information content is high, but it is not easy to get it out, i.e. computational costs for pair-wise identifcation of associated SNPs
\item previous studies: pathway, network, SNP restriction (but how to restrict?)
\item what is lacking: combination and the possibility to allow/investigate possibly new interactions
\item Tested our method on PD data
\item Results: interaction classes perform very good
\end{itemize}





\section{Material and Methods}
%What we did
%
%How was the problem studied?
%    Include detailed information, so that a knowledgeable
%         reader can reproduce the experiment
%    Extensive detail can be described in the supplementary
%         material

\begin{itemize}
\item Short desciption of the PD GWAS
\item SNP data structure: Explanation of the use and combinded data
\item SNP set building (background, why we build them like that)
\item jSLAT explanation
\item application to PD
\end{itemize}

%%%%%%%%%%%%%%%%%%%%%%%%%%%%%%%%%%%%%%%%%%%%%%%%%%%%%%%%%%%%%%%%%%%%%%%%%%%%%%%%%%%%%%%%
%%%%%%%%%%%%%%%%%%%%%%%%%%% BEGIN FIGURE 2 %%%%%%%%%%%%%%%%%%%%%%%%%%%%%%%%%%%%%%%%%%%%%
\begin{figure}[t!h]
\centering \includegraphics[width=0.8\columnwidth]{Overview.jpg}
\caption{Method...}
\label{fig:Ceramide}
\end{figure}
%%%%%%%%%%%%%%%%%%%%%%%%%%% END FIGURE 2 %%%%%%%%%%%%%%%%%%%%%%%%%%%%%%%%%%%%%%%%%%%%%%%
%%%%%%%%%%%%%%%%%%%%%%%%%%%%%%%%%%%%%%%%%%%%%%%%%%%%%%%%%%%%%%%%%%%%%%%%%%%%%%%%%%%%%%%%


%%%%%%%%%%%%%%%%%%%%%%%%%%%%%%%%%%%%%%%%%%%%%%%%%%%%%%%%%%%%%%%%%%%%%%%%%%%%%%%%%%%%%%%%
%%%%%%%%%%%%%%%%%%%%%%%%%%% BEGIN FIGURE 2 %%%%%%%%%%%%%%%%%%%%%%%%%%%%%%%%%%%%%%%%%%%%%
\begin{figure}[t!h]
\centering \includegraphics[width=0.8\columnwidth]{DataStructure.jpg}
\caption{SNP data structure}
\label{fig:Ceramide}
\end{figure}
%%%%%%%%%%%%%%%%%%%%%%%%%%% END FIGURE 2 %%%%%%%%%%%%%%%%%%%%%%%%%%%%%%%%%%%%%%%%%%%%%%%
%%%%%%%%%%%%%%%%%%%%%%%%%%%%%%%%%%%%%%%%%%%%%%%%%%%%%%%%%%%%%%%%%%%%%%%%%%%%%%%%%%%%%%%%


\section{Results}
%What have we found out?
%
%Only representative results, essential for the Discussion,
%    should be presented
%Add supplementary materials for data of secondary
%    importance
%Do not attempt to "hide" data in the hope of saving it for a
%    later paper.

\begin{itemize}
\item Table with percentage of interaction sets
\item Results of PD
\item Programm + link to homepage
\end{itemize}




\section{Discussion}
%(Short summary: What did you find out?)
%
%What did you learn from your study?
%
%What inferences can be drawn from the findings?
%
%What are the practical and theoretical implications of your
%        findings?
%
%How do your results relate previous research?
%
% (Which problems are next to be solved?)
%
%%%%%%%%%%%%%%%%%%%%%%%%%%%%%%%%%%%%%%%%%%%%%%%%%%%%%%%%%%%%%%%%%%%%%%%%%%%%%
%
 Problem: GWAS is expensive keeps lots of information which is propbaly not used\\
 1. Furhter meta-analyses, pathway, interaction\\
 2. Our method is perfect because it combines all aspects of previous research\\
 3. Combines known and predicted interaction information => known and new information in this study\\
 4. makes the investigation smaller but augment the information content\\
 5. The best sets are interaction sets, shows that the method combines the best SNPs (see diagram)\\
 6. Comparison with literature, esp. known PD genes/SNPs\\
 7. Results are transparent and can be repeated (see programm on the following page..\\





\section{Conclusion}
%How the work advances the filed from the present state of
%     knowledge
%
%Without clear Conclusions, reviewers and readers will find it
%     difficult to judge the work, and whether or not it merits
%     publication in the journal
%
%Do not repeat the abstract or just list experimental
%     results!!!

We advance the field with a new method that augement the use of GWAS.\\
It was necessary to use a new method combining pathway and interaction data\\
We solved the problem of the extreme computational power, by resticting with a good biological background the set combination.\\



\section*{Acknowledgements}
\paragraph{Funding:}
Federal Ministry of Education and Research (BMBF, Germany) in the National
Genome Research Network (NGFN+) under grant number 01GS08134.
\paragraph{Conflict of interest:} None declared.

\bibliography{document}
%\begin{thebibliography}{document.bib}
%
%
%\bibitem[CE1]{clareke}
%Clarke, F., Ekeland, I.:
%Nonlinear oscillations and boundary-value problems for
%Hamiltonian systems.
%Arch. Rat. Mech. Anal. {\bfseries 78} (1982) 315--333
%.
%.
%\end{thebibliography}


\end{document} 